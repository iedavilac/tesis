% Todos mis shorcuts para escribir más rápido.

\DeclareMathOperator{\Ran}{Ran}
\DeclareMathOperator{\Ker}{Ker}
\DeclareMathOperator{\im}{Im}

% Nuevos entornos


\newtheorem{teor}{Teorema}
\newcommand{\teo}[2]{
	\begin{teor}[\textbf{#1}]		% Comienza el entorno teorema
	#2
	\end{teor}
	}

% Nuevos paquetes

\usepackage{amssymb} % Para cargar \mathfrak

% Nuevos operadores matemáticos

\DeclareMathOperator{\Tr}{Tr}

\newcommand{\imp}{\implies}			% Simbolo de implicacion
\newcommand{\supr}[1]{\underset{#1}{\sup}}
\newcommand{\mbf}[1]{\mathbf{#1}}     % Negrita en modo matemático
\newcommand{\lra}{\leftrightarrow}     % Flecha derecha e izquierda
\newcommand{\h}{\hat}             % hat para operadores
\newcommand{\red}[1]{\color{red}{#1}}
\newcommand{\green}[1]{\color{green}{#1}}
\newcommand{\blue}[1]{\color{blue}{#1}}
\newcommand{\pr}{\partial}      % Abreviacion para \partial
\newcommand{\cd}{\cdot}           % \cdot
\newcommand{\cds}{\cdots}           % \cdots
\newcommand{\inceq}{\subseteq}    % incluido e igual
\newcommand{\vc}[1]{\vec{#1}}     % vector
\newcommand{\dg}{^\dagger}
\newcommand{\conj}[1]{#1^*}       % conjugado
\newcommand{\pescalar}[2]{#1\cd #2}    % Producto escalar
\newcommand{\f}[2]{\frac{#1}{#2}}           % Short version for \frac
\newcommand{\dott}[1]{\overset{\cdot\cdot}{#1}} % Doble Punto encima (dt)
\newcommand{\nab}{\nabla}     % Shortcut for nabla
%\newcommand{\eval}{\big\rvert}  % Raya vertical para indicar evaluación
%\newcommand{\deg}[1]{#1^{\circ}}    % Grados
\newcommand{\la}{\leftarrow}        % Leftarrow
\newcommand{\mc}[1]{\mathcal{#1}}      % Tipografia caligrafia
\newcommand{\mf}[1]{\mathfrak{#1}}      % Tipografia frakture (gótico)
\newcommand{\ms}[1]{\mathscr{#1}}		% Cursiva
\newcommand{\tf}{\therefore }			% Los tres puntitos en triangulo
\newcommand{\sder}[2]{\frac{d #1}{d #2}} % Derivada simple de #1 respecto a #2
\newcommand{\der}[3]{\frac{d^{#1}#2}{d #3^{#1}}}  % Derivada n-sima de #1 respecto a #2
\newcommand{\sparc}[2]{\frac{\partial #1}{\partial #2}} %Derivadas parciales
\newcommand{\parc}[3]{\frac{\partial^{#1}#2}{\partial #3^{#1}}} %Derivada parcial n-esima respecto de #3
\newcommand{\m}[1]{\mathbb{#1}}	% Hace una letra R --> \mathbb{R}
\newcommand{\inc}{\subset}   % Incluido
\newcommand{\ndvec}[2]{(#1_1,#1_2,\ldots,#1_{#2})} %Crea un vector #2-dimensional con nombre #1
\newcommand{\ci}{\imath}		% Unidad imaginaria
\newcommand{\ptodo}{\forall}	% Para todo simbolo
\newcommand{\me}[1]{#1\m Z}		% Multiplos enteros de #1: #1Z.
\newcommand{\tq}{\mid}			% Simbolo para tal que...
\newcommand{\pp}[1]{#1^{\prime\prime}\mkern-1.2mu} %#1´´
\newcommand{\e}[1]{e^{#1}}		% Exponencial de #1
\newcommand{\om}{\omega}			% Shortcut para omega
\newcommand{\Om}{\Omega}			% Shortcut para Omega
\newcommand{\lam}{\lambda}          % Lambda
\newcommand{\Lam}{\Lambda}         % Lambda mayuscula
\newcommand{\al}{\alpha}          % alpha
\newcommand{\be}{\beta}           % beta
\newcommand{\gm}{\gamma}         % gamma
\newcommand{\Gm}{\Gamma}          % Gamma
\newcommand{\del}{\delta}         % Delta
\newcommand{\sg}{\sigma}          % Sigma
\newcommand{\Del}{\Delta}
\newcommand{\rel}{\sim}
\newcommand{\uvec}[1]{\bm{\hat{\mathbf{#1}}}}   % Vector unitario
\newcommand{\vct}[1]{\vec{\mathbf{#1}}}
\newcommand{\ra}{\rightarrow}
\newcommand{\eps}{\epsilon}
\newcommand{\ex}{\exists}
\newcommand{\bp}[1]{\left(#1\right)}
\newcommand{\bb}[1]{\left[#1\right]}
\newcommand{\bl}[1]{\left\{#1\right\}}
\newcommand{\deld}[1]{\delta^{(3)}(#1)}      % Delta de Dirac en 3d
\newcommand{\ddrc}[2]{\delta^{(#1)}(#2)}      % Delta de Dirac en Nd
\newcommand{\lrpr}{\overset{\lra}{\pr}}		% left right partial
\newcommand{\slashd}{\kern-0.5em\raise0.22ex\hbox{/}}
\newcommand{\barra}[1]{\cancel{#1}}
