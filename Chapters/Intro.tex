% Introduction and motivation

\chapter{Introducción y Motivación} % Chapter title

\label{ch:intro} % For referencing the chapter elsewhere, use \autoref{ch:examples}

Los solitones juegan un papel muy importante en muchas áreas de la física, desde materia condensada, fluidos, física de partículas y cosmología a incluso teoría de cuerdas. Un tipo importante de solitón es el vórtice, el cuál ha sido observado experimentalmente en superconductores tipo II formando un arreglo conocido como enrejado de Abrikosov.

En el área de la cosmología, las cuerdas cósmicas, que son solitones topológicos unidimensionales hipotéticos que se piensan que pudieron existir en los primeros instantes del universo, se pueden encontrar como soluciones del modelo abeliano de Higgs. De hecho, este modelo es el ejemplo prototípico de teoría cuantica de campo que presenta este sipo de soluciones vorticiales.

En física del estado sólido, en particular en relación con el efecto Hall cuántico fraccional, teorías de campo efectivas que incluyen términos de Chern-Simons han probado ser de gran importancia. Una descripción efectiva de los electrones en un fluido de Hall cuántico superconductor puede ser descrita por una teoría de campos escalar con un término de interacción del tipo Chern-Simons. Estas teorías con términos de Chern-Simons también presentan soluciones vorticiales que son una naturaleza distinta a los vórtices que estudaremos en esta tesis.

Uno de los problemas abiertos más importantes en física de partículas surgió a partir de la introducción de la Cromodinámica Cuántica, el llamado problema del confinamiento de quark o simplemente confinamiento. Este fenómeno es de gran importancia ya que explica porqué no se han detectado quarks libres hasta la fecha. Aún no se ha encontrado una prueba analítica del confinamiento en todas las teorías de gauge no abelianas, y nuevos métodos de cálculo deben ser desarrollados. Estos nuevos métodos deben ser de naturaleza no perturbativa, en donde los solitones juegan un rol importante. Aunque en este trabajo no tocaremos teorías de gauge no abelianas, es importante remarcar que en ciertas teorías de gauge abelianas también exhiben confinamiento.

Avocándonos al tema de este trabajo, los vortices abelianos son solitones topológicos en 2 dimensiones que aparecen como soluciones en el modelo de Higgs abeliano. Este modelo es una teoría de campo gauge $U(1)$ con un potencial cuártico cuya variedad de vacío posee una topología no trivial. Esta topología no trivial provee estabilidad a los vórtices. Estos vórtices llevan una carga topológica que es igual al \emph{winding} del campo de Higgs $\phi$ en el infinito o, de forma equivalente, al número de ceros de $\phi$ contado con multiplicidades. Un tipo especial de estos vórtices son los vórtices autoduales o vórtices críticos los cuales aparecen después de un argumento de Bogomolny sobre el funcional de energía. Estos son soluciones a un sistema de ecuaciones diferenciales de primer orden, las conocidas ecuaciones de Bogomolny, que también satisfacen las ecuaciones de Euler-Lagrange. Estas ecuaciones se pueden juntar en una sóla ecuaciones diferencial de segundo orden llamada ecuación de Taubes sujeta a condiciones de contorno. Esta ecuación no es fácil de resolver y sólo se conocen un puñado de soluciones analíticas, por lo que se suelen resolver numéricamente. Cronológicamente, la primera solución analítica apareció en un paper de 1977 de Edward Witten en el contexto de la teoría de Yang-Mills $SU(2)$. Más tarde, otras dos soluciones explícitas fueron encontradas por Dunajski [Contatto22] usando trascendentes de Painlevé.

Otros cuatro tipos de solitones fueron presentados por Manton usando una generalización de la ecuación de Taubes, los cuales son los vórtices de Popov, Jackiw-Pi, Ambjorn-Olesen y Bradlow. Junto con los vórtices hallados por Witten, se tienen cinco vórtices llamados vórtices exóticos. Los vórtices de Witten y Popov surgen de una reducción de simetría de las ecuaciones de anti autodualidad de Yang-Mills (ASDYM) con grupo de gauge $SU(2)$ y $SU(1,1)$ respectivamente. Es natural preguntarse si los tres restantes vórtices también pueden surgir de una reducción de simetría de las ecuaciones ASDYM. La respuesta a esta pregunta es afirmativa, así como lo descubrieron Contatto y Dunajski, presentamos sus resultados en el capítulo \ref{ch:selfduality}.

Finalizamos este trabajo presentando modelos modificados del modelo de Higgs abeliano. Analizaremos como la introducción de impurezas del tipo magnéticas afecta las soluciones vorticiales. Estudiaremos este efecto tanto para vórtices criticos y no críticos. Estos últimos son estudiados de forma numérica utilizando el lenguaje de programación Python.

Con toda esta introducción y motivación empecemos definiendo el concepto de un solitón topológico de una forma más precisa.


\section{Solitón topológico}

Un solitón es una onda solitaria o excitación de algún campo que se propaga sin deformarse, es decir, manteniendo su forma e intensidad en el tiempo, que además tampoco se deforma cuando colisiona con otros solitones. El interés de este trabajo son solitones que tienen un origen topológico con la diferencia que estos si pueden ser afectados en colisiones, es decir, pueden dispersarse.

Para ilustrar el origen topológico de este tipo de solitones, consideremos la teoría más simple que presenta solitones topológicos: una teoría $1+1$ con un campo escalar con un potencial polinómico cuártico. El solitón respectivo se hace llamar \emph{domain wall}. El respectivo lagrangiano de la teoría es
\begin{align}
	\mc L = \f 12\pr_\mu\phi\pr^\mu\phi-\f{\lambda}4(\phi^2-\chi^2)^2, \ \ \lambda,\chi>0.,
\end{align}
el cuál posee una simetría discreta $\m Z_2$, $\phi\ra-\phi$. El vacío de la teoría se encuentra fácilmente minimiazando el potencial cuártico. El campo escalar puede escoger entre dos vacíos, $\phi=\chi$ o $\phi=-\chi$. Por lo tanto, el vacío rompe esta simetría $\m Z_2\ra\m I$. Una configuración que tenga al vacío $\phi=-\chi$ en $x\ra-\infty$ y al vacío $\phi=\chi$ en $x\ra+\infty$ debe pasar por $\phi=0$ creando una densidad de energía distinta de cero, esto es, una partícula solitónica. En este ejemplo el \emph{domain wall} vive en un espacio de una dimensión espacial y una dimensión temporal y se puede pensar como una pseudopartícula localizada y de energía finita. Es común que este objeto se presente extendido en dos dimensiones espaciales y una temporal. Enfaticemos que la existencia del \emph{domain wall} recae en la topología no trivial de la variedad de vacío $\mc M=\{-\chi,\chi\}$.

En general, este tipo de defecto está caracterizado por el $l$-avo grupo de homotopía $\pi_l(\mc M)$ de la variedad de vacío $\mc M$. Para solitones topológicos viviendo en una teoría en $d+1$ dimensiones espacio temporales uno puede clasificarlos de acuerdo a $l$. Para el \emph{domain wall} se tiene que $l=0$, para $l=1$ tenemos a los vórtices que son solitones topológicos en $d=2$ caracterizados por $\pi_1(\mc M)=\m Z$, para $l=2$ tenemos a los monopolos que viven en un espacio $d=3$, finalmente con $l=3$ tenemos a los instantones.

\section{Teorema de Derrick}

Una pregunta importante acerca de los solitones es sobre su estabilidad. El teorema de Derrick, basado en un argumento de escaleo, nos dice que no hay configuraciones de campo de energía finita en más de una dimensión, a parte del vacío, que presente un punto estacionario. Esto significa que solitones de energía finita no son estables para $d>1$. Repasemos el argumento de Derrick brevemente. Por simplicidad, consideremos una teoría de campo escalar en $d$ dimensiones espaciales y hagamos el siguiente rescaleo $x\ra \mu x$, $\phi(x)\ra\phi(\mu x)$, entonces la derivadas se rescalean como $\pr_i\ra\mu\pr_i$ mientras que el elemento $d^dx\ra \mu^{-d}d^dx$. Asumamos también que $\phi$ describe una solución estacionaria no trivial, es decir, que debe ser un punto crítico de la energía
\begin{align}
	E[\phi(x)]=\int_{\m R^d}d^dx\bb{|\pr_i\phi|^2+V[\phi]}
\end{align}
La energía después del rescaleo es
\begin{align}
	E[\phi(\mu x)] = \int_{\m R^d} d^d(\mu x)\bb{\mu^{2-d}|\pr_i\phi|^2+\mu^{-d}V[\phi(\mu x)]}
\end{align}
Para que esta energía posea un punto estacionario para todo valor de $\mu$, es claro que
\begin{align}
	\f{dE[\phi(\mu x)]}{d\mu}\left|_{\mu=1}\right. =0 \label{eq:0.2.0.3}
\end{align}
donde escogimos el valor arbitrario de $\mu=1$. Desarrollando a ecuación anterior se tiene
\begin{align}
	(2-d)\int_{\m R^d}d^dx|\pr_i\phi(x)|^2=d\int_{\m R^d}d^dxV[\phi(x)]
\end{align}
Esta ecuación nos da dos posibilidades para $d$. Para $d=1$ podemos tener configuraciones no triviales pues las contribuciones del escaleo del término cinético pueden compensar las del potencial (de hecho son iguales ambas contribuciones). Para $d=2$ esto ya no es posible pues $V=0$, caso que corresponden a los \emph{lumps} en el modelo sigma. Armemos a nuestra teoría de campo escalar con un campo de gauge, de modo que cuando rescalemos la derivada covariante lo hagamos del siguiente modo $D_\mu\ra\mu D_\mu$, entonces claramente el campo de gauge rexcalea como $A_\mu\ra\mu A_\mu$, por lo tanto la nueva energía rescaleada es
\begin{align}
	E[\phi(\mu x)] = \int_{\m R^d} d^d(\mu x)\bb{\f{\mu^{4-d}}{g^2}F_{\mu\nu}F^{\mu\nu}+\mu^{2-d}|D_i\phi|^2+\mu^{-d}V[\phi(\mu x)]}
\end{align}
Aplicando \eqref{eq:0.2.0.3} a esta energía llegamos a
\begin{align}
	(4-d)\int_{\m R^d}d^dx\f{1}{g^2}F_{\mu\nu}F^{\mu\nu}+(2-d)\int_{\m R^d}d^dx|D_i\phi|^2 = d\int_{\m R^d}d^dx V[\phi(x)],
\end{align}
claramente vemos que para $d=2$ la contribución de los campos de gauge balancea la contribución del potencial dando pie a la aparición de los vórtices. Un comentario adicional es que para $d=4$ el término de los campos de gauge no escalea y no hay necesidad de término cinético ni de potencial, dando origen a un solitón sin rompimiento espontáneo de la simetría, llamado instantón. Tal vez el solitón más importante en la física de altas energías es el instantón hallado por Belavin, Polyakov, Schwartz y Tyupkin (BPST) en la teoría de Yang-Mills $SU(2)$ pura, el cuál revisaremos en \ref{sec:4.1}.

Pero si no hay potencial, por lo tanto no hay variedad de vacío, entonces, ¿cuál es el origen topológico de los instantones? La respuesta es simplemente que los instantones mapean el borde del espacio tiempo $\pr\m R^4$, que puede ser identificado con la esfera cuatro dimensional $S^3$ mediante una compactificación, al grupo de gauge $SU(2)\equiv S^3$. Tales mapeos son caracterizados por el tercer grupo de homotopía $\pi_3(S^3)=\m Z$. Por lo tanto, el número instantónico o la carga del instanton es de origen topológico.

Otro comentario interesante sobre los instantones es que la acción es inversamente proporcional a la constante de acoplamiento $S\propto g^{-2}$. Las amplitudes de transición (o amplitudes en general) están descritas por la integral de caminos de Feynman, que mediante una rotación de Wick se puede escribir como $\sim e^{-S}$. Por lo tanto, los instantones se vuelven importantes en el régimen de constante de acoplamiento grande, llamado también \emph{strong coupling}.
