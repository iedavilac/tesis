% Chapter X

\chapter{Supersimetría y vórtices} % Chapter title

\label{ch:supervortices} % For referencing the chapter elsewhere, use \autoref{ch:name} 

%----------------------------------------------------------------------------------------

En este capítulo abordaremos la conexión que existe entre supersimetría y vórtices. En especial tocaremos la conexión entre la ecuación de Bogomolny y la condición de la existencia de una extensión supersimétrica del modelo; notada por Schaposnik y De Vega [schapo-devega] y clarificada más tarde por Witten y Olive [wittenolive].

\section{Ecuaciones de Bogomolny y supersimetría}

Ya hemos visto que existe un valor especial de la constante de acoplamiento $\lambda$ para el cual las ecuaciones de movimiento de segundo orden se reducían a ecuaciones diferenciales de primer orden y para el cuál se conoce una solución exacta.

Estas ecuaciones de primer orden son las llamadas ecuaciones de Bogomolny, las cuales no sólo aparecen para el modelo abeliano de Higgs sino en muchas otras teorías. En [schapo-devega] se nota que este \emph{fine-tuning} de la constante de acoplamiento permite una extensión supersimétrica del modelo abeliano de Higgs.

Como motivación, consideremos la versión supersimétrica de una teoría de campo escalar en dos dimensiones
\begin{align}
	S = \int d^2x\bb{\f 12(\pr_\mu\phi)^2+\f i2\bar\psi\gm^\mu\pr_\mu\psi-\f 12V(\phi)^2-\f 12V'(\phi)\bar\psi\psi},
\end{align}
donde $\psi$ es un fermión de Majorana y $V(\phi)$ es la función potencial de la teoría, que por ahora es arbitraria. En dos dimensiones las matrices gama se escriben como
\begin{align}
	\gm^0 = -i\sg^2 = \begin{pmatrix}
	0 & -1 \\ 1 & 0
	\end{pmatrix}, \ \ \gm^1 = \sg^3 = \begin{pmatrix}
	1 & 0\\ 0 & -1
	\end{pmatrix}
\end{align}
Las ecuaciones de movimiento para los campos se obtienen fácilmente y son
\begin{align}
	\Box\phi &= V(\phi)V'(\phi)+\f 12V''(\phi)\bar\psi\psi \label{eq:5.1.0.3}\\
	\pr\slashd\psi &- V'(\phi)\psi =0 \label{eq:5.1.0.4}
\end{align}
Esta acción es invariante (en el sentido que la variación de la acción es a lo mucho una derivada total) ante las siguientes transformaciones de supersimetría
\begin{align}
	\delta\phi &= \bar\eps\psi\\
	\delta\psi &= (\pr\slashd\phi-V(\phi))\eps \label{eq:5.1.0.6}\\
	\delta\bar\psi &= -\bar\eps(\pr\slashd\psi+V(\phi))
\end{align}
y por consiguiente, debido al teorema de Noether, la corriente conservada asociada es
\begin{align}
	J^\mu = -(\pr\slashd\phi+V(\phi))\gm^\mu\psi,
\end{align}
y la supercarga asociada es
\begin{align}
	Q = -\int dx(\pr\slashd\phi+V(\phi))\gm^0\psi
\end{align}
Luego, escribiendo $Q=(Q^+ \ \ Q^-)^T$ es fácil ver que
\begin{align}
	Q^+ &= \int dx(\pr_0\phi\psi^++(\pr_1\phi+V(\phi))\psi^-) \label{eq:5.1.0.10}\\
	Q^- &= \int dx(\pr_0\phi\psi^-+(\pr_1\phi-V(\phi))\psi^+) \label{eq:5.1.0.11}
\end{align}
Estas cargas no cierran totalmente el álgebra supersimétrica, ya que se satisface $Q_+^2 = P_+$, $Q_-^2 = P_-$, donde $P_{\pm}=P_0\pm P_1$ y
\begin{align}
	\{Q_+,Q_-\} = 2\int dxV(\phi)\f{\pr\phi}{\pr x}.
\end{align}
Recordemos que para el álgebra supersimétrica este último término se debe anular. Sin embargo, queda este término de superficie, el cuál también se puede escribir como
\begin{align}
	\{Q_+,Q_-\} = 2\int dx H'(x), \label{eq:5.1.0.13}
\end{align}
donde $H(\phi)$ es una función tal que $H'(\phi)=V(\phi)$. Entonces, el conmutador de las supercargas es la integral de una divergencia total la cuál debería anularse. Pero en un estado solitónico este término no es necesariamente nulo. Por ejemplo, considere $V(\phi)=-\lambda(\phi^2-a^2)$ de modo que el potencial es el típico potencial cuártico de Higgs $\lambda^2(\phi^2-a^2)^2$, el cuál tiene dos vacíos en $\phi=\pm a$. Llamando $T$ al lado derecho de \eqref{eq:5.1.0.13}, vemos que para una configuración de \emph{domain wall} $\phi(-\infty)=-a$ y $\phi(\infty)=a$
\begin{align}
	T = \int_{-\infty}^\infty dx\f{\pr}{\pr x}\bp{2a^2\lambda-\f 23\lambda\phi^3} =\f{4\lambda a^3}3.
\end{align}
Es claro que $T$ está relacionada con la carga topológica $\int dx(\pr\phi/\pr x)$. Esta nueva álgebra modificada $\{Q_+,Q_-\}=T$ tiene muchas consecuencias interesantes. La primera es que permite establecer un límite sobre la masa de una partícula. Considere lo siguiente
\begin{align}
	(Q_+\pm Q_-)^2 &= Q_+^2+Q_-^2\pm(Q_+Q_-+Q_-Q_+) \nonumber\\
				   &= P_++P_-\pm T \label{eq:5.1.0.15}
\end{align}
Pero $(Q_+\pm Q_-)^2\geq 0$, entonces, $P_++P_-\geq |T|$. Para una sola partícula de masa $M$ en reposo, $P_0=M$ y $P_1=0$, entonces $P_+=P_-=M$ y la ecuación anterior implica
\begin{align}
	M\geq |T|.
\end{align}
Ahora nos preguntamos cuando se satura esta desigualdad. Es claro, a partir de ecuación \eqref{eq:5.1.0.15} que la desigualdad se satura para estados $\ket{\al}$ tales que
\begin{align}
	(Q_+\pm Q_-)\ket\al =0.
\end{align}
Clásicamente, esta condición se satisface para todos los estados solitónicos de la teoría. Esto se puede ver fácilmente a partir de ecuaciones \eqref{eq:5.1.0.10} y \eqref{eq:5.1.0.11}, que implican para una configuración estática $\pr\phi/\pr t=0$
\begin{align}
	(\pr_1\phi+V(\phi))\psi^-=0 \ \ \ \text{o} \ \ \ (\pr_1\phi-V(\phi))\psi^+=0 \label{eq:5.1.0.18}
\end{align}
que no son más que las ecuaciones de Bogomolny para esta teoría. En efecto, la energía de una configuración estática es
\begin{align}
	M = \int dx\f 12\bp{(\pr_1\phi)^2+V^2(\phi)}
\end{align}
la cuál se puede escribir, mediante el argumento de Bogomolny, de la siguiente forma
\begin{align}
	M = \f 12\int dx\bp{(\pr_1\phi)^2\pm V(\phi)^2}^2\mp T
\end{align}
del cuál se pueden derivar fácilmente ecuaciones \eqref{eq:5.1.0.18} cuando se satura la desigualdad.

Otra cosa que podemos notar es que la transformación supersimétrica para el fermión $\psi$ \eqref{eq:5.1.0.6} se escribe, para una configuración estática, como sigue
\begin{align}
	\delta\psi = \begin{pmatrix}
	\pr_1\phi-V(\phi) & 0\\
	0 & -\pr_1\phi- V(\phi)
	\end{pmatrix}\begin{pmatrix}
	\eps^+ \\ \eps^-
	\end{pmatrix}.
\end{align}
Haciendo $\delta\psi=0$ es otra forma de obtener las ecuaciones de Bogomolny sin necesidad de construir la corriente y luego las cargas. Además, para cada ecuación de Bogomolny está asociada una solución de la ecuación de Dirac
\begin{align}
	\psi_0 = (\pr\slashd\phi-V)\begin{pmatrix}
	\psi_0^+ \\ 0
	\end{pmatrix} \ \ \ \text{o} \ \ \ \psi_0 = (\pr\slashd\phi+V)\begin{pmatrix}
	0 \\ \psi_0^-
	\end{pmatrix} 
\end{align}
Uno puede entender la aparición de estos modos cero notando que un \emph{domain wall} junto con $\psi=0$ resuelven las ecuaciones de movimiento \eqref{eq:5.1.0.3} y \eqref{eq:5.1.0.4} del modelo. Pero una transformación de supersimetría de esta solución seguirá siendo una solución, lo que significa que $\delta\psi$ satisface la ecuación de Dirac.


\section{El modelo supersimétrico $N=1$ en $2+1$}

Comencemos introduciendo una versión supersimétrica del modelo abeliano de Higgs. Definimos la métrica como $g_{\mu\nu}=diag(+,-,-)$. El campo de Higgs $\phi$ junto con el higgsino $\psi$ y un campo auxiliar $F$ son componentes del supercampo escalar complejo $\Phi(y,\theta)$, que en $2+1$ se escribe como
\begin{align}
	\Phi(y;\theta) &= \phi(y)+\sqrt 2\theta\psi(y)-\theta\theta F(y) \label{eq:5.2.0.1} \\
	\Phi\dg(y;\bar\theta) &= \phi^*(y)+\sqrt 2\overline{\theta\psi}(y)-\overline{\theta\theta} F(y),
\end{align}
donde $y^\mu=x^\mu+i\theta\sg^\mu\bar\theta$ y $\theta^2 = \theta\theta$. En $2+1$ un supercampo escalar $\mc N=1$ no puede ser quiral [Alexandre2003]. Definimos la superderivada covariante como
\begin{align}
	\mc D_\al = \f{\pr}{\pr\theta^\al}+i(\sg^\mu\bar\theta)_\al\pr_\mu, \ \ \ \overline{\mc D_\al} = \f{\pr}{\pr\overline\theta^\al}+i(\theta\sg^\mu)_\al\pr_\mu,
\end{align}
son las superderivadas covariantes en la variable $x^\mu$. Para algunas cuentas necesitaremos escribirlas en términos de $y^\mu$ como
\begin{align}
	\mc D_\al = \f{\pr}{\pr\theta^\al}+2i(\sg^\mu\bar\theta)_\al\f{\pr}{\pr y^\mu}, \ \ \ \overline{\mc D_\al} = \f{\pr}{\pr\overline\theta^\al}
\end{align}

El campo de gauge $A_\mu(x)$ abeliano junto con el fotino $\lambda(x)$ se acomodan en el supercampo vectorial $V$, que escrito en el gauge de Wess-Zumino es
\begin{align}
	V_{WZ} = \theta\sg^\mu\bar\theta A_\mu(x)+i\theta^2\overline{\theta\lambda}(x)-i\bar\theta^2\theta\lambda(x)
\end{align}
Finalmente, también introducimos un supercampo escalar real, necesario para implementar el mecanismo de ruptura de simetría de Fayet-Illiopoulos [Fayet-Illiopoulos], que se escribe como
\begin{align}
	S(x;\theta) = N(y)+\sqrt 2\theta\chi(y)+\f 12\theta^2 D(y)
\end{align}
y el supercampo spinorial lo escribimos como
\begin{align}
	W_\al = -\f 14 \overline{\mc D}^2\mc D_\al V_{WZ} = -i\lambda_\al(y)+i(\sg^{\mu\nu}\theta)_\al F_{\mu\nu}(y)+\theta^2(\sg^\mu\pr_\mu\bar\lambda(y))_\al
\end{align}
El lagrangiano supersimétrico para el modelo abeliano de Higgs $\mc N=1$, escrito en el formalismo de superespacio, tiene la forma
\begin{align}
	L_{\mc N=1} = \int d^2\theta\bp{\f 12 W^\al W_\al+\sqrt{2\lambda}S\Phi\dg\Phi}+\int d^2\theta d^2\bar\theta \bp{\Phi\dg e^{V_{WZ}}\Phi+S\dg S+\eta S}
\end{align}
donde $\eta$ es el parámetro de Fayet-Iliopoulos. Escribiendo el lagrangiano en términos de sus componentes bosónicos y fermiónicos nos da
\begin{align}
	L_{\mc N=1} &= -\f 14 F_{\mu\nu}F^{\mu\nu}+\f 12\pr^\mu N\pr_\mu N+\f 12(D^\mu\phi)\dg D_\mu\phi+\f 12D^2+\sqrt{2\lambda} D|\phi|^2 \nonumber\\
	&\eta D+\f 12|F|^2+\sqrt{2\lambda}N(F^*\phi+F\phi^*)+\f i2\lambda\pr\slashd\bar\lambda+\f i2\chi\pr\slashd\bar\chi+\f i2\psi D\slashd\bar\psi-\nonumber\\
	&\sqrt{2\lambda}N\bar\psi\psi+\f i2(\bar\psi\lambda\phi-\bar\lambda\psi\phi^*)-\sqrt{2\lambda}(\bar\psi\chi\phi+\bar\chi\psi\phi^*) \label{eq:5.2.0.11}
\end{align}
Podemos eliminar los campos auxiliares $F$ y $D$ insertando en el lagrangiano anterior sus ecuaciones de movimiento correspondientes
\begin{align}
	F &= -2\sqrt{2\lambda} N\phi,\\
	D &= -\sqrt{2\lambda}|\phi|^2-\eta.
\end{align}
El rompimiento de la simetría se da cuanto el término $D$ se anula para cierto valor del campo $\phi_0$. Tal valor se halla fácilmente como
\begin{align}
	\phi_0^2 = -\f{\eta}{\sqrt{2\lambda}}.
\end{align}
Reemplazando los campos auxiliares en el lagrangiano \eqref{eq:5.2.0.11} obtenemos
\begin{align}
	L_{\mc N=1} &= -\f 14 F_{\mu\nu}F^{\mu\nu}+\f 12\pr^\mu N\pr_\mu N+\f 12(D^\mu)^* D_\mu\phi-4\lambda N^2|\phi|^2-\lambda (|\phi|^2-\phi_0^2)^2\nonumber\\
	&+\f i2\bar\lambda\pr\slashd\lambda+\f i2\bar\chi\pr\slashd\chi+\f i2\bar\psi D\slashd\psi-\sqrt{2\lambda}N\bar\psi\psi\nonumber\\
	&+\f i2(\bar\psi\lambda\phi-\bar\lambda\psi\phi^*)-\sqrt{2\lambda}(\bar\psi\chi\phi+\bar\chi\psi\phi^*) \label{eq:5.2.0.15}
\end{align}
Nótese que este lagrangiano se reduce al lagrangiano del modelo abeliano de Higgs si "apagamos" los campos fermiónicos, es decir, anulamos los campos $\lambda,\psi$ y $\chi$, y  si hacemos también $N=0$. Incluso aparece el típico término del potencial cuártico sin necesidad de agregar algún superpotencial, ya que este nace de pedir la ruptura de la simetría a través del término de Fayet-Iliopoulos junto con el término $D$.

Ahora nos gustaría hacer que el lagrangiano anterior posea una supersimetría extendidad $\mc N=2$. Para ver como podemos hacer esto, daremos un repaso brevísimo sobre ciertos aspectos de teorías de gauge supersimétricas $\mc N=2$ en tres dimensiones.

\subsection{Aspectos generales de teorías $\mc N=2$ en $3d$}

En tres dimensiones existen 4 cargas supersimétricas [Csaba-Martone,Aharony-Hanany-Seiberg-Strassler,Gates-Grisaru-Roczek], a diferencia de las 8 supercargas en cuatro dimensiones. Estas se obtienen a partir de una reducción de dimensión del álgebra $\mc N=1$ en $4d$:
\begin{align}
	\{Q_\al,Q_\be\} = \{\overline Q_\al,\overline Q_\be\} =0, \ \ \{Q_\al,\overline Q_\be\} = 2(\sg^i)_{\al\be}P_i+2i\eps_{\al\be}Z
\end{align}
donde $\sg^{i=0,1,2}=(i\sg_2,\sg_3,\sg_1)$ y $Z$ es la carga central y corresponde a la componente $P_3$ del momento en $4d$. Una diferencia con las matrices gama en 4d es que estas son todas reales. En cuatro dimensiones, los espinores transformaban en la representación fundamental del grupo de Lorentz $SL(2,\m C)$. En tres dimensiones, el grupo de Lorentz es $SL(2,\m R)$ que posee 3 generadores (2 boosts y una rotación), y los espinores formar spinores de Majorana de dos componentes reales.

Las coordenadas en el superespacio ahora son $z = (x^\mu,\theta_\al,\theta_\be)$, donde $\mu=0,1,2$ y $\theta_{\al,\be}$ son variables de Grassmann reales. En $4d$, los supercampos quirales se escriben de igual forma que \eqref{eq:5.2.0.1}, con la diferencia que en $3d$ el espinor de Weyl $\psi$ se descompone en dos fermiones de Majorana reales.

El supercampo vectorial $\mc N=2$ en $3d$ se escribe como
\begin{align}
	V = i\theta\bar\theta\sg+i\theta\sg^i\bar\theta A_i+i\theta^2\bar\theta\lambda-i\bar\theta^2\theta\lambda+\f 12\theta^2\bar\theta^2 D, \label{eq:5.2.1.2}
\end{align}
donde se ha separado explícitamente el campo de gauge en $3d$ $A_i$ del campo de gauge escalar que surge de la reducción dimensional $\sg\sim A_3$. Nótese que a diferencia del supercampo vectorial de la teoría $3d$ $\mc N=1$ de la sección anterior, en $\mc N=2$ el supercampo contiene un campo escalar real $\sg$ que puede adquirir un valor de expectación de vacío distinto de cero que puede rompe la simetría de gauge de la teoría. Sin embargo, en $3d$ no es el único campo escalar que puede adquirir un valor de expectación distinto de cero. En tres dimensiones el dual de Hodge del tensor de campo electromagnético, $\star F$, es una $1$-forma. Esta $1$-forma la podemos escribir como el diferencial de algún campo escalar $\gm$
\begin{align}
	\star F = d\gm.
\end{align}
Este campo escalar se conoce como el fotón dual y es análogo al campo de gauge $A$ cuando $F=dA$ (con la diferencia que $A$ es una 1-forma y por lo tanto, $F$ es una 2-forma). Debido a la regla de cuantización de Dirac, $\gm$ debe ser una función periódica y entonces tomar valores en $S^1$. Uno puede combinar ambos campos escalares ($\sg$ y $\gm$) en un sólo campo escalar
\begin{align}
	\tilde\phi = \sg+i\gm.
\end{align}
Definamos $\mc J = d\gm$. Como $d^2=0$, entonces queda claro que $d\mc J=0$. Este resultado no es más que una consecuencia de las ecuaciones de Maxwell; por lo tanto, podemos considerar a $\mc J$ como una corriente conservada. Esta corriente conservada está asociada a alguna simetría global de la teoría. Claramente, esta simetría es $U(1)_{\mc j}$ debido a como definimos el tensor de campos. En SUSY, corrientes conservadas se pueden integrar dentro de un supercampo $\Sigma$ que es lineal, es decir, que satisface $\mc D^2\Sigma=\overline{\mc D}^2\Sigma=0$. Un supercampo que satisface las restricciones anteriores tiene [Bertolini] la siguiente expansión en componentes
\begin{align}
	\Sigma = -\f i2\overline{\mc D}^\al\mc D_\al V = \sg+\theta\bar\lambda+\overline\theta\lambda+\f 12\theta\sg^i\overline\theta\pr_i\gm+i\theta\overline\theta D+\f i2\overline\theta^2\theta\sg^i\pr_i\lambda-\nonumber\\
	\f i2\theta^2\overline\theta\sg^i\pr_i\overline\theta+\f 14\theta^2\overline\theta^2\pr^2\sg.
\end{align}


Ahora, a partir de estos supercampos nos gustaría construir lagrangianos en $3d$ con supersimetría $\mc N=2$. A partir del supercampo vectorial \eqref{eq:5.2.1.2} podemos definir el supercampo electromagnético $W_\al = -\f 14\overline{\mc D}^2\mc D_\al V$ y con esto construir la parte electromagnética de la teoría
\begin{align}
	L_{YM} &= \int d^2\theta W_\al W^\al \label{eq:5.2.1.6} \\
	&= -\f 14 F_{ij}F^{ij}+ D_i\sg D^i\sg+D^2+\overline\lambda\sg^iD_i\lambda
\end{align}
Alternativamente, uno puede usar el supercampo lineal $\Sigma$ para construir una acción análoga a la anterior
\begin{align}
	L_{YM} = \int d^2\theta d^2\overline\theta \Sigma^2
\end{align}
De hecho ambos lagrangianos son iguales.

En $3d$, la acción de Yang-Mills(en el caso no abeliano) \eqref{eq:5.2.1.6} no es la única combinación que es invariante de gauge que se puede agregar a la acción. De hecho uno puede agregar un término de Chern-Simons que en una toería abeliana se escribe como
\begin{align}
	L_{CS} = \int d^2\theta d^2\overline\theta \Sigma V.
\end{align}
Aunque no estudiaremos teorías con términos de Chern-Simons, creemos que vale la pena mencionarlos por su mero interés. Referimos a [Seiberg-Intrilligator] para una \emph{review} excelente. A parte de estos términos también podemos agregar términos de Fayet-Iliopoulos como pasa en $4d$.

Para agregar materia y dinámica a la teoría debemos acoplar los supercampos escalares con el supercampo vectorial. Para campos quirales el lagrangiano se escribe como
\begin{align}
	L = \int d^2\theta d^2\overline\theta Q e^{V}Q\dg
\end{align}
donde $Q = \phi_Q+\theta\psi+\theta^2 F$ es el campo quiral. Escrito en componentes, el lagrangiano es
\begin{align}
	L = |D_i\phi_Q|^2+\sg^2|\phi_Q|^2+D|\phi_Q|^2+i\overline\psi\sg^iD_i\psi-i\sg\overline\psi\psi+i(\phi_Q\dg\overline\lambda\psi-\overline\psi\lambda\phi_Q)+|F|^2
\end{align}
Con todos estos elementos escribamos el lagrangiano más general de una teoría en $3d$ con supersimetría $\mc N=2$ (sin término de Chern-Simons):
\begin{align}
	L_{\mc N=2} &= \int d^2\theta W_\al W^\al+\int d^2\theta d^2\overline\theta\overline Qe^{2gV}Q+\eta\int d^2\theta d^2\overline\theta V\\
	&= -\f 14 F_{\mu\nu}F^{\mu\nu}+\f 12\pr_\mu\sg\pr^\mu\sg+i\overline\lambda\pr\slashd\lambda+\f 12D^2+g\eta D+gD|\phi_Q|^2+D^\mu\phi_Q\dg D_\mu\phi_Q+\nonumber\\
	&i\overline\psi D\slashd\psi+g^2\sg^2|\phi_Q|^2+g\sg\overline\psi\psi+\sqrt{2}g(\overline\lambda\psi\phi_Q\dg-\overline\psi\lambda\phi_Q)+|F|^2
\end{align}
Este lagrangiano no es más que la reducción dimensional de un lagrangiano $\mc N=1$ en $4d$. La reducción de dimensión se obtiene al hacer que los campos no dependan de alguna de las coordenadas y cierta componente del campo de gauge se transforma en un campo escalar real. En este caso, hicimos que los campos no dependan de $x^2$ y la componente $A_2\ra \sg$.

Eliminando los campos auxiliares, obtenemos la forma final para el lagrangiano $\mc N=2$ en $3d$:
\begin{align}
	L_{\mc N=2} &= -\f 14 F_{\mu\nu}F^{\mu\nu}+\f 12\pr_\mu\sg\pr^\mu\sg+i\overline\lambda\pr\slashd\lambda+D^\mu\phi_Q\dg D_\mu\phi_Q+\f{g^2}2(|\phi|^2+\eta)^2\nonumber\\
	&i\overline\psi D\slashd\psi+g^2\sg^2|\phi_Q|^2+g\sg\overline\psi\psi+\sqrt{2}g(\overline\lambda\psi\phi_Q\dg-\overline\psi\lambda\phi_Q) \label{eq:5.2.1.14}
\end{align}
donde hemos reescaleado el supercampo vectorial $V\ra 2gV$ donde $g$ es el equivalente a la carga electrica $e$. La acción \eqref{eq:5.2.1.14} ya ha sido derivada por ejemplo en [Ivanov91] partiendo de una acción $\mc N=2$ supersimétrica y luego escribiéndola en términno de supercampos $\mc N=1$. Este procedimiento es el inverso a lo que queremos llegar. Para obtener una condición sobre la constante de acoplamiento $\lambda$ debemos partir de una acción $\mc N=1$ y llegar a una acción $\mc N=2$.


\subsection{De $\mc N=1$ a $\mc N=2$}

Ahora que tenemos la forma general de los lagrangianos $\mc N=1$ y $\mc N=2$, podemos plantearnos que condiciones deben satisfacer los parámetros del lagrangiano $\mc N=1$ para que podamos extender esta simetría a $\mc N=2$.  Primero para poder extender la supersimetría $\mc N=1$ a $\mc N=2$ debemos doblar el número de generadores de supersimetría. Esto significa aumentar el número de campos presentes en la teoría. Contemos el número de grados de libertad fermiónicos y bosónicos de los campos. Tanto $S,\Phi$ como $\Gamma_\al$ tiene 1 grado de libertad fermiónico y bosónicos. En efecto, $\Phi$ tiene al campos escalar $\phi$ y al fermión $\psi_\al$; de igual manera, el supercampop vectorial $\Gamma_\al$ tiene al fermión $\rho_\al$ y a $A_\mu$ que en $3d$ aporta sólo un grado de libertad (polarización); lo mismo con el supercampo escalar $S$. Por lo tanto, en una teoría $\mc N=2$, podemos construir un supercampo que contenga todos estos grados de libertad. Con este fin, vamos a juntar los fermiones $\chi$ y $\rho$ en un sólo fermión de Dirac
\begin{align}
	\Sigma = \chi-i\rho.
\end{align}
Ahora, queremos encontrar un campo escalar que contenga a $\Sigma$ como su grado de libertad fermiónico. Es fácil ver que tal campo es [Alexandre2003]
\begin{align}
	\Lambda = S+iD^\al \Gamma_\al = N+\theta\Sigma+\f 12\theta^2 D+i\theta\sg^\mu\sg^\nu\theta\pr_\mu A_\nu
\end{align}
El término cinético para este campo es
\begin{align}
	\f 12\int d^2\theta D^\al\Lambda D_\al\overline\Lambda = -\f 14 F_{\mu\nu}F^{\mu\nu}+\f i2\overline\Sigma\pr\slashd\Sigma+\f 12\pr_\mu N\pr^\mu N+\f{D^2}2+\nonumber\\
	\f 12(\pr_\mu A^\mu)^2+\f 12\pr_\mu(\overline\Sigma\sg^\mu\Sigma),\label{eq:5.2.2.3}
\end{align}
donde el último término es un término de borde y puede desecharse. El penúltimo término puede eliminarse si se impone la condición de gauge $\pr_\mu A^\mu=0$. Esta condición de gauge no es arbitraria, de hecho, es equivalente a pedir que el supercampo vectorial satisfaga $D^\al\Gm_\al=0$, de modo que pueda contener una corriente conservada. Esta condición de gauge fue encontrada en [Alexandre[5]] y también la vimos en la sección anterior cuando construimos el camo lineal $\Sigma$. Entonces, $A_\mu$ tiene el rol de una corriente topológica conservada. Entonces, eliminando el término de borde y aplicando la condición de gauge en \eqref{eq:5.2.2.3} llegamos a una expresión para el lagrangiano $\mc N=2$ en términos de los supercampos $\mc N=1$
\begin{align}
	L_{\mc N=2,gauge} &= \int d^2\theta D^\al\Lambda D_\al\overline\Lambda\nonumber\\
	&=-\f 14 F_{\mu\nu}F^{\mu\nu}+\f i2\overline\Sigma\pr\slashd\Sigma+\f 12\pr_\mu N\pr^\mu N+\f{D^2}2
\end{align}
Incluimos materia en la teoría de forma usual mediante un supercampo escalar complejo $\mc N=1$, $Q$:
\begin{align}
	Q = \phi+\theta\psi+\f 12\theta^2 F.
\end{align}
El término cinético para este campo es claramente
\begin{align}
	D^\al QD_\al\overline Q\vert_{\theta^2} = \pr_\mu\phi\pr^\mu\phi+i\overline\psi\pr\slashd\psi+|F|^2+\text{términos de borde},
\end{align}
El acoplamiento de $Q$ con el supercampo vectorial se obtiene a partir de los siguientes términos
\begin{align}
	D^\al Q\Gm_\al\overline Q\vert_{\theta^2} &= -\f 12(\overline\phi\psi\lambda-\overline A_\mu\pr^\mu\phi+i\overline\psi A\slashd\psi)\\
	SQ\overline Q\vert_{\theta^2} &= \f 12(D|\phi|^2-\phi\overline\psi\chi-\overline\phi\psi\chi+N\phi\overline F+N\overline\phi F-N\overline\psi\psi)\\
	Q\Gm^\al\Gm_\al\overline Q\vert_{\theta^2} &= -|\phi|^2A^\mu A_\mu,
\end{align}
de modo que el lagrangiano para la materia es
\begin{align}
	L_{\mc N=2,materia} &= \f 12\int d^2\theta\bb{(D^\al-ig\Gm^\al)Q(D_\al+ig\Gm_\al)\overline Q+2gSQ\overline Q}\nonumber\\
	&= \f i2\overline\psi D\slashd\psi+\f 12 D_\mu\phi D^\mu\phi-\f g2(\phi\overline{\psi\Sigma}+\overline\phi\psi\Sigma)-\f g2N\overline\psi\psi\nonumber\\
	&+\f g2 D|\phi|^2+\f 12|F|^2+\f g2N(\phi\overline F+\overline\phi F), \label{eq:5.2.2.10}
\end{align}
donde $g$ es la constante de acoplamiento y $D_\mu=\pr_\mu+ig A_\mu$. Este lagrangiano lo derivamos en la sección anterior a partir de una reducción dimensional de una teoría $\mc N=1$ en $4d$. En esa sección, no elevamos un lagrangiano $\mc N=1$ a $\mc N=2$ por lo que no se obtuvo ninguna condición sobre la constante de acoplamiento.

Para obtener el potencial cuártico del modelo de Higgs abeliano, eliminemos los campos auxiliares
\begin{align}
	\overline F+gN\overline\phi &=0\\
	D+\f g2\phi\overline\phi &=0
\end{align}
de modo que los términos en \eqref{eq:5.2.2.10} que contienen campos auxiliares se convierten en
\begin{align}
	(L_{\mc N=2,gauge}+L_{\mc N=2,materia})_{aux.} &= \f 12D^2+\f g2D|\phi|^2+\f 12|F|^2+\f g2 N(\overline\phi F+\overline F\phi)\\
	&= -\f{g^2}2N^2|\phi|^2-\f{g^2}8|\phi|^4.
\end{align}
Agreguemos también un término de Fayet-Iliopoulos de la forma
\begin{align}
	L_{F.I.} = -\f{g^2}2\phi_0^2\int d^2\theta(\Lambda+\overline\Lambda)=-\f{g}2\phi_0^2 D.
\end{align}
donde $\phi_0$ es una constante real. La adición de este término al lagrangiano modifica la ecuación de movimiento para $D$ como
\begin{align}
	D+\f{g}2|\phi|^2-\f g2\phi_0^2 =0
\end{align}
de modo que el potencial cuártico se obtiene al eliminar los campos auxiliares
\begin{align}
	(L_{\mc N=2,gauge}+L_{\mc N=2,materia}+L_{F.I.})_{aux.} = -\f{g^2}2N^2|\phi|^2-\f{g^2}{8}(|\phi|^2-\phi_0^2)^2
\end{align}
Es fácil ver que la constante de acoplamiento de Higgs es $\lambda=\f{g^2}8$, la cual fue encontrada por [Schaposnik] como una condición de consistencia para pasar de $\mc N=1$ a $\mc N=2$.